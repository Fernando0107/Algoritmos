\documentclass[a4paper, 11pt]{article}
\usepackage{comment} % enables the use of multi-line comments (\ifx \fi) 
\usepackage{lipsum} %This package just generates Lorem Ipsum filler text. 
\usepackage{fullpage} % changes the margin
\usepackage{natbib}
\usepackage{graphicx}
\usepackage[spanish]{babel}
\usepackage[]{algorithm2e}
\usepackage{placeins}
\usepackage{float}
\usepackage[utf8]{inputenc}
\usepackage{algorithmic}

\begin{document}
%Header-Make sure you update this information!!!!
\noindent
\large\textbf{Laboratorio \# 5} \hfill \textbf{Juan Diego Sique Martínez} \\
\normalsize Algoritmia y Complejidad \hfill \textbf{Fernando González} \\
Catedrático Ernesto Rodríguez \hfill \\
Auxiliar Juan Roberto Alvarado \hfill 

\section*{Problema}
Realizar un algoritmo que determine el subconjunto de elementos que dentro de un arreglo, al ser sumados consecutivamente den como resultado 0.

\section*{Algoritmo propuesto}
\begin{algorithm}[H]
 \KwData{Un arreglo {\bf A}}
 \KwResult{Un subconjunto del arreglo original cuyos elementos sumados dan como resultado 0}
 negativos = []\;
 positivos = []\;
 combinacionesPositivos = []\;
 combinacionesNegativos = []\; 
 respuesta = []\;
 separarPositivosNegativos(A, positivos, negativos)\;
 respuesta = compararIdenticos(positivos, negativos)\;
 \If{ respuesta != []
 }{return respuesta}
 combinacionesPositivos = combinaciones(positivos)\;
 combinacionesNegativos = combinaciones(Negativos)\;
 respuesta = compararIdenticos(combinacionesPositivos, combinacionesNegativos)\;
return respuesta
 \caption{Laboratorio 5}
\end{algorithm}

\section*{¿Qué hace el método <<separarPositivosNegativos>>?}
Esta función recibe como parámetro dos arreglos, uno donde colocará los números positivos, y otro donde pondrá los números negativos. De manera iterativa recorre el arreglo, destinando los elementos según si son mayores o menores que cero.

\begin{algorithm}[H]
 \KwData{Tres arreglos {\bf A B C}}
 \KwResult{Los elementos de un conjunto clasificados según su procedencia o anterioridad al cero.}
 \For{ i = 0 to A.lenght
 }{
 \eIf{A[i] $>$ 0}{
 B.append(A[i])}{C.append(A[i])}
 }
 \caption{Laboratorio 5}
\end{algorithm}


\section*{¿Qué hace el método <<CompararIdenticos>>?}
El método realiza una comparación entre elementos idénticos entre dos listas, si se cumple esta condición rompe el ciclo y retorna un subarreglo con los elementos reversos.

\begin{algorithm}[H]
 \KwData{Dos arreglos {\bf A B}}
 \KwResult{Un arreglo con elementos reversos}
 \For{ itemA in A
 }{
 \For{itemB in B}{
 \If{itemA == -(itemB)}{ return [itemA, itemB]}
}
 }
 \caption{Laboratorio 5}
\end{algorithm}


\section*{¿Qué hace el método <<Combinaciones>>?}
El método de combinaciones obtiene todas las posibles combinaciones entre elementos de un conjunto.

\end{document}
